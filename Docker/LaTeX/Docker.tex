
\documentclass[11pt]{scrartcl}
\usepackage{graphicx}
\usepackage{amsmath, amssymb, textcomp}
\usepackage[utf8]{inputenc}
\usepackage[english]{babel}
\usepackage{ucs}
\usepackage[T1]{fontenc}
\usepackage{subcaption} 	
\usepackage{hyperref}
\usepackage{fullpage}	
\usepackage{color}
\usepackage{enumitem}
\setlist[itemize]{noitemsep, topsep=1pt}
\usepackage[font=footnotesize]{caption}
\usepackage{draftwatermark}
\SetWatermarkText{Martin Fané}
\SetWatermarkScale{3}
\SetWatermarkColor[gray]{0.99}
\usepackage[numindex,nottoc,section]{} 


\def\Reg{\textsuperscript{\textregistered}}
\def\TM{\textsuperscript{\texttrademark}}
\def\CR{\textsuperscript{\textcopyright}}	


\usepackage{listings, xcolor}
\lstset
{	
%	language     = Docker,
	basicstyle   = \ttfamily,
	keywordstyle = \color{blue}\ttfamily,
	stringstyle  = \color{orange}\ttfamily,
	commentstyle = \color{green}\ttfamily,
	morecomment  = [l][\color{teal}]{\#},
%	backgroundcolor = \color{lightgray}\ttfamily,
	breaklines=true,
	tabsize=3,
%	numbers=left,
	numberstyle=\color{black},
	numbersep=5pt, 
	frame=single
}

\begin{document}
\tableofcontents

\newpage
\section{Basics} \label{basics}
\begin{itemize}
	\item Docker: Set of Platform as a Service (\texttt{PaaS}) product delivering software in packages called \texttt{containers}
	\item Architecture:
	\begin{itemize}
		\item Client-server architecture
		\item Docker client talks to the \texttt{Docker daemon}, which builds, runs, and distributes the Docker containers
		\item Docker client and daemon communicate using \texttt{REST API}
		\item \texttt{Docker Compose} is another client to work with applications consisting of a set of containers
	\end{itemize}
	\item Docker daemon (\texttt{dockerd}) 
	\begin{itemize}
		\item manages \texttt{images, containers, networks,} and \texttt{volumes}
		\item can communicate with other daemons to manage Docker services
	\end{itemize}
	\item Docker client (\texttt{docker}) 
	\begin{itemize}
		\item sends commands, e.g. \texttt{docker run} to \texttt{dockerd}, which carries them out
		\item uses Docker API
		\item can communicate with more than one daemon
	\end{itemize}
	\item Docker registries stores Docker \texttt{images}
\end{itemize}

\subsection{Docker objects} \label{docker_objects}
\subsubsection{Images} \label{images}
\begin{itemize}
	\item Read-only (immutable) template with instructions for creating a Docker \texttt{container}
	\item Often based on another \texttt{image} with additional customization
	\item Built from an instructional file named \texttt{Dockerfile}
	\item Each instruction in an \texttt{Dockerfile} creates a \texttt{layer} in the \texttt{image}
	\item Each \texttt{layer} represents a set of file system changes that add, remove, or modify files
	\item Only changed \texttt{layers} will be rebuild when changing the \texttt{Dockerfile} and rebuilding the \texttt{image}
\end{itemize}

\subsubsection{Containers} \label{containers}
\begin{itemize}
	\item Runnable instance of an \texttt{image}
	\item Can be connected to one or more networks
	\item Isolated from other \texttt{containers} and its host machine
	\item Defined by its \texttt{image}
	\item Include all of the dependencies required for an app to work
\end{itemize}

\subsubsection{Dockerfile} \label{dockerfile}
\begin{itemize}
	\item Provide the instructions needed to build an \texttt{image} which is then the recipe for a container
	\item Are written by the user whereas an \texttt{image} is written by the machine based on the \texttt{Dockerfile}
	\item Syntax:
	\begin{lstlisting}
FROM <image>:<tag>
RUN <install some depedencies>
CMD <command that is executed on `docker container run`>
	\end{lstlisting}
\end{itemize}


\subsection{Docker CLI basics} \label{cli_basics}
\begin{itemize}
	\item Docker Engine is made up of \texttt{CLI} client, a \texttt{REST API}, and a \texttt{Docker daemon}
	\item When running \texttt{docker container run}, the CLI sends a request through REST API to the Docker daemon which takes care of images, containers and other resources
	\item List all containers:\begin{lstlisting}	
docker container ls    # list only running containers
docker container ls -a # list all containers
docker container ls -a | grep <expr> # filtering for <expr>
		
docker ps (-a) 		  # shorter form
	\end{lstlisting}
	\item Remove containers:
	\begin{lstlisting}
docker container rm <id1> <id2> <id3>
docker container prune # deletes all stopped containers
	\end{lstlisting}
	\item Remove images:
	\begin{lstlisting}
docker image rm <image>
docker image prune # removes unnamed and not used images
	\end{lstlisting}
\end{itemize}

\subsubsection{Most used commands} \label{commands}
\begin{tabular}{|l|l|l|}
	\hline
	\textbf{Command}							&	\textbf{Description}					&	\textbf{Shorthand}		\\
	\hline
	\texttt{docker image ls} 					&	Lists all images						&	\texttt{docker images}	\\
	\hline
	\texttt{docker image rm <image>}			&	Removes all images						&	\texttt{docker rmi}		\\
	\hline
	\texttt{docker image pull <image>}			&	Pulls an image from a docker registry	&	\texttt{docker pull}	\\
	\hline
	\texttt{docker container ls -a}				&	Lists all containers					& 	\texttt{docker ps -a}	\\
	\hline
	\texttt{docker container run <image>}		&	Runs a container from an image			&	\texttt{docker run}		\\
	\hline
	\texttt{docker container rm <container>}	&	Removes a container						&	\texttt{docker rm}		\\
	\hline
	\texttt{docker container stop <container>}	&	Stops a container						&	\texttt{docker stop}	\\
	\hline
	\texttt{docker container exec <container>}	&	Executes a command inside a container	& 	\texttt{docker exec}	\\
	\hline
\end{tabular}


\end{document}