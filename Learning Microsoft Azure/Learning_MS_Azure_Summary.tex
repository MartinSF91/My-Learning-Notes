
\documentclass[11pt]{scrartcl}
\usepackage{graphicx}
\usepackage{amsmath, amssymb, textcomp}
\usepackage[utf8]{inputenc}
\usepackage[english]{babel}
\usepackage{ucs}
\usepackage[T1]{fontenc}
\usepackage{subcaption} 		% Subfigures
\usepackage{hyperref}
\usepackage{fullpage}	
\usepackage{color}
\usepackage{enumitem}
\setlist[itemize]{noitemsep, topsep=1pt}
\usepackage[font=footnotesize]{caption}
\usepackage{draftwatermark}
\SetWatermarkText{Martin Fané}
\SetWatermarkScale{3}
\SetWatermarkColor[gray]{0.99}
\usepackage[numindex,nottoc,section]{} 


\begin{document}
\section*{Cloud Fundamentals}
\begin{itemize}
	\item Infraastructure as a Service:
	\begin{itemize}
		\item Computing deployment category in which the cloud provider delivers infrastructure through the cloud
		\item Delivery of IT infrastructure resources like web servers, DB servers, compute storage, networking, computing data centers as service
		\item Buy pizza at the store and bake it at home
	\end{itemize}
	\item Platform as a Service:
	\begin{itemize}
		\item Cloud model where users can create, build, and deploy applications on the cloud without worrying about IT infrastructure behind it
		\item Provides computing services, development, and monitoring tools for application development
		\item Provider takes care of physical infrastructures, data centers, hardware, OS
		\item User only needs to write and deploy application code on the platform
		\item Pizza delivery
	\end{itemize}
	\item Software as a Service:
	\begin{itemize}
		\item Software-on-demand cloud model where cloud provider give access to a fully developed application
		\item Enables users to access and use applications online without installation
		\item Accessible through web browser or servers
		\item Pay for the service on a subscription basis
		\item Order pizza at a restaurant
	\end{itemize}
	\item Containers as a Service:
	\begin{itemize}
		\item Deploy applications in  containers (containerization)
		\item Container: runtime that contains essential computing resources needed to run an application, including the core part of the host OS (kernel) and its shared resources like storage across a host
	\end{itemize}
	\item Serverless:
	\begin{itemize}
		\item Backend services are provided by a cloud service provider
		\item Third-party provider manages the infrastructure and automatically provisions and scales resources as needed
		\item Provider handles server infrastructure, OS and other low-level components $\to$ no need to manage underlying infrastructure
	\end{itemize}
\end{itemize}


\section*{MS Azure Fundamentals}
\subsection*{Services}
\begin{itemize}
	\item Compute Services:
	\begin{itemize}
		\item Provide quickly available and on-demand resources like OS, networking, disks, processors, and memory
		\item Enables to build web and mobile applications, deploy and manage VMs, build apps in containers in the cloud, create batch jobs, etc.
	\end{itemize}
	\item Core Azure Storage Services:
	\begin{itemize}
		\item AZ Blobs: Store scalable binary data, text, or Data Lake Storage Gen2 big data analytics
		\item AZ Files: Fully manageable file shares for deployments on-premise or cloud. Accessible anywhere through Server Message Block (SMB) protocol
		\item AZ Queues: Collect large messages that are accessed via authenticated HTTP calls
		\item AZ Managed Disks: Store block-level volumes for AZ vmS
	\end{itemize}
	\item Core Azure DB Services:
	\begin{itemize}
		\item AZ SQL DB: Cloud-hosted SQL databases that are fully managed, intelligent, and secure
		\item AZ Cosmos DB: Create and migrate NoSQL workloads to the cloud like Cassandra, MongoDB, and
		other NoSQL databases
		\item AZ Cache for Redis DB: Build fast and scalable applications with Redis in-memory data store
		\item AZ DB for PostgreSQL, MySQL, and MariaDB: Create fully managed and scalable databases for PostgresSQL, MySQL, and MariaDB
		\item AZ SQL Edge: Build IoT edge-optimized SQL database engine with built-in AI
	\end{itemize}
\end{itemize}

\subsection*{Core Architecture and Resource Management Concepts}
\begin{itemize}
	\item Management groups:
	\begin{itemize}
		\item Top level of the core structure
		\item Administrators manage everything about user access, compliance, and policies for subscriptions
		\item Subscriptions within a management group automatically inherit settings, conditions, and restrictions added in the group
		\item AZ RBAC for all resources and role definitions is supported in the root management group
	\end{itemize}
	\item Azure subscriptions:
	\begin{itemize}
		\item Are like a big container for all user accounts and resources they have accessed or used within the subscription
		\item Every subscription has a limit of resources that a certain user can create and use
		\item Use subscriptions to control monthly bill and resource costs
	\end{itemize}
	\item Resource groups:
	\begin{itemize}
		\item Group services or resources using resource groups
		\item Acts as a logical container where resources like servers, web apps, DBs, storage, and monitoring are deployed, managed, and stored
	\end{itemize}
	\item Resources:
	\begin{itemize}
		\item DBs, servers, VMs, or web apps
		\item All resources or services must be added to a resource group, which acts like a logical container
	\end{itemize}
	\item Resource Manager:
	\begin{itemize}
		\item Management and deployment service that provides users the capability to add, edit, and delete resources in AZ
		\item By using ARM, an organization can manage user access control and organize resources securely even after deployment
		\item ARM templates are commonly used to automate deployments and implement infrastructure as Code
		\item IaC creates a great advantage, nd enables deployment automation of the infrastructure in the cloud
		\item Using IaC, you can automate deployments by generating templates for the same environment every time
	\end{itemize}
\end{itemize}
	
	
\subsection*{User Identities, Roles, and Active Directories}
\begin{itemize}
	\item Role-Based Access Control:
	\begin{itemize}
		\item Helps in authorization and user access management of resources 
		\item Management of identity using RBAC helps in controlling what users can do and cannot do
	\end{itemize}
	\item Roles:
	\begin{itemize}
		\item Security principal:
		\begin{itemize}
			\item Object that represents a security identity that can be authenticated and authorized to access resources
			\item Used to grant ar deny resource permissions
			\item Can be authenticated as user, security group, or process 
			\item When assigning roles to a security principal, you're granting or denying permissions to access specific resources in AZ
		\end{itemize}
		\item Role definition:
		\begin{itemize}
			\item Sets permissions for users or security principals to utilize resources
			\item Each role definition has a set of access controls, or actions, which helps determine which resource activities are permissable
		\end{itemize}
		\item Scope:
		\begin{itemize}
			\item Determines the level at which the role assignment applies
			\item Defines the set of resources the role assignment applies to and can be set at various levels in resource hierarchy (management group, subscription, resource group, and individual resource level)
		\end{itemize}
	\end{itemize}
\end{itemize}
	
	
	
	
	
	
	
	
	
	
	
	
	
	
	
	
	
	
	
	
	
	
	
	
	
	
	
	
	
	
\end{document}