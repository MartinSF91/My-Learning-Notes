
\documentclass[11pt]{scrartcl}
\usepackage{graphicx}
\usepackage{amsmath, amssymb, textcomp}
\usepackage[utf8]{inputenc}
\usepackage[english]{babel}
\usepackage{ucs}
\usepackage[T1]{fontenc}
\usepackage{subcaption} 		% Subfigures
\usepackage{hyperref}
\usepackage{fullpage}	
\usepackage{color}
\usepackage{enumitem}
\setlist[itemize]{noitemsep, topsep=1pt}
\usepackage[font=footnotesize]{caption}
\usepackage{draftwatermark}
\SetWatermarkText{Martin Fané}
\SetWatermarkScale{3}
\SetWatermarkColor[gray]{0.99}
\usepackage[numindex,nottoc,section]{} 


\begin{document}
\subsection*{Introduction}
\begin{itemize}
	\item AI Engineering:
	\begin{itemize}
		\item Focuses less on modeling and training, and more on model adaption and evaluation
		\item Works with models that are bigger and consume more compute resources, and incur higher latency than traditional ML models
	\end{itemize}
	\item Token: 
	\begin{itemize}
		\item Basic unit of a language model (can be a word, character, or part of a word)
		\item Allow models to break words into meaningful components
		\item There are fewer unique tokens than unique words
		\item Number of tokens in a model's dataset isn't the same as its number of training tokens $\to$ the number of training tokens measures the tokens that the model is trained on
	\end{itemize}
	\item Masked language model: Trained to predict a token anywhere in a sequence, using the context from both before and after the missing token $\to$ fill in the blank
	\item Autoregressive language model: 
	\begin{itemize}
		\item Trained to predict the next token in a sequence, using only the preceding tokens
		\item Can continually generate one token after another
	\end{itemize}
	\item Generative models can generate open-ended outputs
	\item Classical ML models are closed-ended since they only can outputs that are among predefined values
	\item Completions are predictions based on probabilities, and not guaranteed to be correct
	\item Self-supervision:
	\begin{itemize}
		\item Model infers labels from the input data
		\item Language modeling is self-supervised because each input sequence provides both the labels (tokens to be predicted) and the contexts the model can use to predict these labels $\to$ labels are inferred from the input data
	\end{itemize}
	\item Foundation model: Can be built upon for different needs
	\item Multimodal model: Models that can work with more than one data modality
	\item Agents: AI that can plan and use tools
	\item Three layers of AI stack:
	\begin{itemize}
		\item Application development: Provide a model with good prompts and necessary context
		\item Model development: Tooling for developing models, including frameworks for modeling, training, finetuning, and inference optimization
		\item Infrastructure: Tooling for model infrastructure, including tooling for model serving, managing data and compute, and monitoring
	\end{itemize}
	\item Inference optimization: Make models faster and cheaper
	\item Prompt engineering: Get AI models to express the desirable behaviors from the input alone, without changing the model weights
\end{itemize}

\subsection*{Foundation Models}
\subsubsection*{Sampling}
\begin{itemize}
	\item Process of constructing output
	\item How does a model chooses an output from all possible options
	\item Makes an AI's output probabilistic
	\item To generate the next token, the model first computes the probability distribution over all tokens in the vocabulary
	\item Greedy sampling: Always pick the most likely outcome $\to$ creates boring outputs since the most common words are responded
	\item Instead of picking the most likely token, the model can sample the next tokens according to the probability distribution
	\item Sampling strategies:
	\begin{itemize}
		\item Temperature: A higher temperature reduces the probabilities of common tokens and increases the probability of rarer tokens
		\item Top-k: Pick top-k logits to perform softmax over these $\to$ smaler k makes the text more predictable but less interesting
		\item Top-p: 
		\begin{itemize}
			\item Allows for a more dynamic selection of values to be sampled from than top-k
			\item Model sums the probabilities of the most likely next values in descending order and stops when the sum reaches p $\to$ only the values within the cumulative probability are considered
			\item Focuses only on the set of most relevant values for each context and allows outputs to be more contextually appropriate
			\item Typically range from 0.95 to 0.99
		\end{itemize}
	\end{itemize}
	\item Logit vector:
	\begin{itemize}
		\item Corresponds to one possible value $\to$ one token in the model's vocabulary
		\item Logit vector size is the size of the vocabulary
		\item Do not represent probabilities
		\item To convert logits to probabilities, a softmax layer is often used
	\end{itemize}
	\item Sparsity allows for more efficient data storage and computation (large percentage of zero-value parameters)
	\item FLOP: Unit for a model's compute requirement $\to$ measures the number of floating point operations performed for a certain task
	\item Utilization: Measures how much of the maximum compute capacity one can use
	\item Post-Training:
	\begin{itemize}
		\item Issues of a pre-trained model: 
		\begin{itemize}
			\item Self-supervision optimizes the model for text completion, not conversations
			\item Outputs can be biased or wrong
		\end{itemize}
		\item Solution:
		\begin{itemize}
			\item Supervised finetuning (SFT): finetune the pre-trained model on high-quality instruction data to optimize models for conversations instead of completion
			\item Preference finetuning: Finetune the model to output responses that aligh with human preferences
		\end{itemize}
		\item Preference finetuning is typically done with reinforcement learning or reinforcement learning from human feedback (RLHF)
		\item Pre-training optimizes token-level quality
		\item Post-training optimizes the model to generate responses that users prefer
	\end{itemize}
	\item RLHF:
	\begin{itemize}
		\item Train a reward model that scores the foundation model's output
		\item Optimize the foundation model to generate responses for which the reward model will give maximal scores
	\end{itemize}
	\item Reward model: 
	\begin{itemize}
		\item Outputs a score for how good the response is
		\item Generate multiple responses for each prompt and the resulting data is comparison data with the format (prompt, winning\_response, losing\_response)
	\end{itemize}
	\item Robustness: A model is robust if it doesn't dramatically change its outputs with small variations in the input
	\item Inconsistency: Generating very different responses for the same or slightly different prompts
	\item Hallucination:
	\begin{itemize}
		\item Response that isn't grounded in facts
		\item Cannot differentiate between the data it's given and the data it generates
		\item Snowball hallucinations: Model continues ti justify the initial wrong assumption
		\item Also caused by the mismatch between the model's internal knowledge and the labeler's internal knowledge
	\end{itemize}
\end{itemize}


\subsection*{Evaluation Methodology}
\subsubsection*{Metrics}
\begin{itemize}
	\item Most autoregressive LMs are trained using cross entropy or perplexity (predictive accuracy metrics $\to$ the more accurately, the lower the metrics)
	\item Entropy:
	\begin{itemize}
		\item Measures how much information, on average, a token carries $\to$ the higher the entropy, the more information
		\item Measures how difficult it is to predict what comes next in a language $\to$ the lower the language's entropy, the more predictable that language
	\end{itemize}
	\item Cross Entropy on a dataset:
	\begin{itemize}
		\item When training a LM on a dataset, the goal is to get the model learn the distribution of this training data
		\item Measures how difficult it is for the LM to predict what comes next in this dataset
		\item An LM is trained to minimize its cross entropy with respect to the training data
		\item A model's cross entropy is its approximation of the entropy of its training data
	\end{itemize}
	\item Bits-per-byte (BPB): Number of bits an LM needs to represent one byte of the original training data
	\item Perplexity: 
	\begin{itemize}
		\item Exponential of entropy and cross entropy
		\item Rules:
		\begin{itemize}
			\item More structured data gives lower expected perplexity
			\item The bigger the vocabulary, the higher the perplexity
			\item The longer the context, the lower the perplexity
		\end{itemize}
	\end{itemize}
\end{itemize}

\subsubsection*{Exact Evaluation}
\begin{itemize}
	\item Judgment without ambiguity
	\item Approaches:
	\begin{itemize}
		\item Functional Correctness: Evaluating a system based on whether it performs the intended functionality
		\item Similarity Measurements Against Reference Data: Evaluate AI's output against reference data
	\end{itemize}
	\item Ground truth: Reference responses
	\item Similarity measurement: Asking an evaluator, exact match, lexical similarity, semantic similarity
	\item Exact match: Generated response matches one of the reference responses exactly
	\item Lexical similarity: 
	\begin{itemize}
		\item Measures how much two texts overlap $\to$ Counting how many tokens two texts have in common
		\item Approximate string matching (fuzzy matching): Measures similarity between two texts by counting how many edits it'd need to convert from one text to another (edit distance)
	\end{itemize}
	\item Semantic similarity: Aims to compute the similarity in semantics, requiring embeddings
	\item Embeddings: Numerical representation (vector) that aims to capture the meaning of the original data
\end{itemize}



\subsubsection*{AI as Judge}
\begin{itemize}
	\item Using AI to evaluate AI
	\item Approaches:
	\begin{itemize}
		\item Evaluate data by itself
		\item Compare a generated response to a reference response
		\item Compare two generated responses and determine which one is better
	\end{itemize}
	\item Challenges:
	\begin{itemize}
		\item AI as a judge criteria aren't standardized
		\item AIs are subjective $\to$ Evaluation results can change based on the judge model and prompt
		\item Probabilistic nature of AI makes it seem too unreliable to act as evaluator
		\item Inconsistency: Same judge, on the same input, can output different scores if prompted differently
		\item Criteria ambiguity: Misinterpret and misuse judge metrics since they are not standardized
		\item Evaluation should be fixed, but AI judges also change over time
		\item Self-bias: Model favors own response over response from other models
		\item first-position-bias: AI judge my favor the first answer in a pairwise comparison
		\item Verbosity bias: Favoring lengthier answers, regardless of their quality
	\end{itemize}
	\item Judge models:
	\begin{itemize}
		\item Reward model: Takes in (prompt, response) pair and scores how good the response is given the prompt
		\item Reference-based judge: Evaluates the generated response with respect to one or more references
		\item Preference model: Takes in (prompt, response 1, response 2) as input and outputs which of the two responses is better for the given prompt
	\end{itemize}
\end{itemize}

\end{document}