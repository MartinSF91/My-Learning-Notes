
\documentclass[11pt]{scrartcl}
\usepackage{graphicx}
\usepackage{amsmath, amssymb, textcomp}
\usepackage[utf8]{inputenc}
\usepackage[english]{babel}
\usepackage{ucs}
\usepackage[T1]{fontenc}
\usepackage{subcaption} 		% Subfigures
\usepackage{hyperref}
\usepackage{fullpage}	
\usepackage{color}
\usepackage{enumitem}
\setlist[itemize]{noitemsep, topsep=1pt}
\usepackage[font=footnotesize]{caption}
\usepackage{draftwatermark}
\SetWatermarkText{Martin Fané}
\SetWatermarkScale{3}
\SetWatermarkColor[gray]{0.99}
\usepackage[numindex,nottoc,section]{} 


\begin{document}
\subsection*{Introduction}
\begin{itemize}
	\item AI Engineering:
	\begin{itemize}
		\item Focuses less on modeling and training, and more on model adaption and evaluation
		\item Works with models that are bigger and consume more compute resources, and incur higher latency than traditional ML models
	\end{itemize}
	\item Token: 
	\begin{itemize}
		\item Basic unit of a language model (can be a word, character, or part of a word)
		\item Allow models to break words into meaningful components
		\item There are fewer unique tokens than unique words
	\end{itemize}
	\item Masked language model: Trained to predict a token anywhere in a sequence, using the context from both before and after the missing token $\to$ fill in the blank
	\item Autoregressive language model: 
	\begin{itemize}
		\item Trained to predict the next token in a sequence, using only the preceding tokens
		\item Can continually generate one token after another
	\end{itemize}
	\item Generative models can generate open-ended outputs
	\item Classical ML models are closed-ended since they only can outputs that are among predefined values
	\item Completions are predictions based on probabilities, and not guaranteed to be correct
	\item Self-supervision:
	\begin{itemize}
		\item Model infers labels from the input data
		\item Language modeling is self-supervised because each input sequence provides both the labels (tokens to be predicted) and the contexts the model can use to predict these labels $\to$ labels are inferred from the input data
	\end{itemize}
	\item Foundation model: Can be built upon for different needs
	\item Multimodal model: Models that can work with more than one data modality
	\item Agents: AI that can plan and use tools
	\item Three layers of AI stack:
	\begin{itemize}
		\item Application development: Provide a model with good prompts and necessary context
		\item Model development: Tooling for developing models, including frameworks for modeling, training, finetuning, and inference optimization
		\item Infrastructure: Tooling for model infrastructure, including tooling for model serving, managing data and compute, and monitoring
	\end{itemize}
	\item Inference optimization: Make models faster and cheaper
	\item Prompt engineering: Get AI models to express the desirable behaviors from the input alone, without changing the model weights
\end{itemize}

\subsection*{Foundation Models}
\begin{itemize}
	\item Sampling: 
	\begin{itemize}
		\item Process of constructing output
		\item How does a model chooses an output from all possible options
		\item Makes an AI's output probabilistic
		\item To generate the next token, the model first computes the probability distribution over all tokens in the vocabulary
		\item Greedy sampling: Always pick the most likely outcome $\to$ creates boring outputs since the most common words are responded
		\item Instead of picking the most likely token, the model can sample the next tokens according to the probability distribution
	\end{itemize}
	\item Sampling strategies:
	\begin{itemize}
		\item Temperature: A higher temperature reduces the probabilities of common tokens and increases the probability of rarer tokens
		\item Top-k: Pick top-k logits to perform softmax over these $\to$ smaler k makes the text more predictable but less interesting
		\item Top-p: 
		\begin{itemize}
			\item Allows for a more dynamic selection of values to be sampled from than top-k
			\item Model sums the probabilities of the most likely next values in descending order and stops when the sum reaches p $\to$ only the values within the cumulative probability are considered
			\item Focuses only on the set of most relevant values for each context and allows outputs to be more contextually appropriate
			\item Typically range from 0.95 to 0.99
		\end{itemize}
	\end{itemize}
	\item Logit vector:
	\begin{itemize}
		\item Corresponds to one possible value $\to$ one token in the model's vocabulary
		\item Logit vector size is the size of the vocabulary
		\item Do not represent probabilities
		\item To convert logits to probabilities, a softmax layer is often used
	\end{itemize}
	\item Sparsity allows for more efficient data storage and computation (large percentage of zero-value parameters)
	\item Number of tokens in a model's dataset isn't the same as its number of training tokens $\to$ the number of training tokens measures the tokens that the model is trained on
	\item FLOP: Unit for a model's compute requirement $\to$ measures the number of floating point operations performed for a certain task
	\item Utilization: Measures how much of the maximum compute capacity one can use
	\item Post-Training:
	\begin{itemize}
		\item Issues of a pre-trained model: 
		\begin{itemize}
			\item Self-supervision optimizes the model for text completion, not conversations
			\item Outputs can be biased or wrong
		\end{itemize}
		\item Solution:
		\begin{itemize}
			\item Supervised finetuning (SFT): finetune the pre-trained model on high-quality instruction data to optimize models for conversations instead of completion
			\item Preference finetuning: Finetune the model to output responses that aligh with human preferences
		\end{itemize}
		\item Preference finetuning is typically done with reinforcement learning or reinforcement learning from human feedback (RLHF)
		\item Pre-training optimizes token-level quality
		\item Post-training optimizes the model to generate responses that users prefer
	\end{itemize}
	\item RLHF:
	\begin{itemize}
		\item Train a reward model that scores the foundation model's output
		\item Optimize the foundation model to generate responses for which the reward model will give maximal scores
	\end{itemize}
	\item Reward model: 
	\begin{itemize}
		\item Outputs a score for how good the response is
		\item Generate multiple responses for each prompt and the resulting data is comparison data with the format (prompt, winning\_response, losing\_response)
	\end{itemize}
	\item Robustness: A model is robust if it doesn't dramatically change its outputs with small variations in the input
	\item Inconsistency: Generating very different responses for the same or slightly different prompts
	\item Hallucination:
	\begin{itemize}
		\item Response that isn't grounded in facts
		\item Cannot differentiate between the data it's given and the data it generates
		\item Snowball hallucinations: Model continues ti justify the initial wrong assumption
		\item Also caused by the mismatch between the model's internal knowledge and the labeler's internal knowledge
	\end{itemize}
\end{itemize}

\end{document}